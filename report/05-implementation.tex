\chapter{Технологическая часть}

В данном разделе будет описан и обоснован выбор средств реализации программного обеспечения и представлены детали реализации.

\section{Средства реализации программного обеспечения}

В качестве языка программирования был выбран язык c++ в силу следующих причин:

\begin{itemize}
	\item имеется опыт разработки на данном языке;
	\item средствами языка можно реализовать все алгоритмы, выбранные в результате проектирования.
\end{itemize}

Так же использовались следующие библиотеки:

\begin{itemize}
	\item кроссплатформленная библиотека Qt для разработки графического интерфейса, поскольку имеется опыт разработки, с использованием этой библиотеки;
	\item библиотека Intel® Threading Building Blocks (Intel® TBB), предназначенная для параллелизма на уровне инструкций во время выполнения для эффективного использования ресурсов процессора \cite{tbb}. Использовалась как средство реализации параллельной версии алгоритма обратной трассировки лучей. Библиотека совместима в другими библиотеками потоков \cite{tbb}, из чего следует, что конфликта с потоками библиотеки Qt не будет.
	\item библиотека stl\_reader, состоящая для одного заголовочного файла и предназначенная для чтения stl файлов, а также преобразования содержимого в пользовательские контейнеры \cite{stlreader}. Необходима в связи с надобностью, исходя из аналитической части, чтения stl-файлов заранее подготовленных  моделей шахматных фигур.
\end{itemize}

Для ускорения сборки программного обеспечивания использовалась утилита cmake \cite{cmake}.

\section{Структура программного обеспечивания}

В разработанном программном обеспечении реализованы следующие классы.

\includeimage  
{classes}
{f}
{H}
{\textwidth}
{Диаграмма классов (ч. 1)}

\includeimage  
{command}
{f}
{H}
{0.6\textwidth}
{Диаграмма классов (ч. 2)}

Ниже приведено писание следующих основных классов, используемых в программном обеспечении.

\begin{itemize}
	\item классы, связанные с объектами сцены;
	\begin{itemize}
		\item Object -- базовый класс объекта сцены, определяет интерфейс объекта;
		\item Box -- описывает параллелепипед, используется для создания «клетки» шахматной доски;
		\item Triangle -- описывает плоскость, ограниченную треугольником, единица декомпозиции модели;
		\item Model -- базовый класс модели сцены;
		\item Camera -- наблюдатель (камера);
		\item LightSource -- точечный источник света;
	\end{itemize}
	\item классы, связанные с использованием модификации обратной трассировки;
	\begin{itemize}
		\item KDTree -- класс, используемый для построения KD-дерева;
		\item BoundingBox -- параллелепипед, использующийся в модификации для определения наличия пересечения луча, исходящего из камеры, с объектом;
		\item Material -- используется как контейнер спектральных характеристик модели;
		\item Ray -- хранит параметрическое описание прямой;
		\item Vector3D -- описывает вектор в декартовых координатах;
	\end{itemize}
	\item классы, связанные с взаимодействием интерфейса и объектов сцены;
	\begin{itemize}
		\item Scene -- класс, описывающий объекты сцены, в одном из методов реализован основной алгоритм обратной трассировки лучей;
		\item SceneManager -- класс, используемый для инкапсулирования сцены;
		\item Facade, *Command, SceneCommand - реализация паттернов команда и фасад для обеспечения интерфейса сцены в совокупности;
		\item QtDrawer - класс, используемый для заполнения объекта QPixmap -- экземпляр класса библиотеки Qt для работы с изображениями \cite{qtdoc},
	\end{itemize}
\end{itemize}

\section{Интерфейс пользователя}

Интерфейс пользователя включает в себя следующие объекты:

\begin{itemize}
	\item область изображения сцены;
	\item список выбора шахматных фигур;
	\item списки выбора цвета и соответствующих координат при добавлении фигуры;
	\item списки выбора координат фигуры при её удалении;
	\item вкладки «Материал поверхности» и «Точечный источник света» для изменения спектральных характеристик соответствующих объектов;
	\item поля «Показатель качества полировки» (целое число, варьируется от 0 до 1000) и «Коэффициент диффузного отражения» (десятичная дробь, два знака после запятой, варьируется от 0.3 до 1);
	\item поле «Выбор интенсивности» для изменения интенсивности точечного источника освещения;
	\item Кнопка «Камера-Источник света» для выбора объекта, местоположение которого изменяется;
	\item Поля «dx», «dy», «dz» для выполнения переноса выбранного объекта (целые числа, варьируются -5000 до 5000) и поля «Ox», «Oy», «Oz» для выполнения поворота вокруг соответствующих осей (в случае выбора «Источника света» вкладка «Поворот» недоступная для заполнения).
\end{itemize}

На рисунке \ref{img:interface} представлен интерфейс пользователя. 

\includeimage  
{interface}
{f}
{H}
{\textwidth}
{Интерфейс пользователя}

\section*{Вывод}

В данном разделе была описана и обоснован выбор средств реализации программного обеспечения и представлены детали реализации.

