\chapter{Исследовательская часть}

В данном разделе будет описана цель исследования по времени, технические характеристики устройства, на котором будет проводиться исследование по времени, а также приведено само исследование.

\section{Технические характеристики устройства}

Ниже приведены технические характеристики устройства, на котором проводилось тестирование:

\begin{itemize}
	\item Операционная система: Ubuntu Mantic Minotaur (development branch);
	\item Оперативная память: 16 ГБ;
	\item Процессор: Intel® Core™ i7-10510U × 8.
\end{itemize}

Тестирование проводилось на ноутбуке, включенном в сеть электропитания. 
Во время тестирования ноутбук был нагружен только встроенными приложениями окружения, окружением, а также непосредственно системой тестирования. 

Использовался компилятор g++. 
Оптимизация отключалась с помощью использования опции компилятора $-O0$.

\section{Постановка цели исследования}

В реализации алгоритма обратной трассировки лучей присутствует рекурсивная часть. В ходе выполнения программы, данная часть выполняется в том случае, если коэффициент отражения поверхности больше 0. В противном случае будет выполняться только линейная часть программы. 

Цель исследования заключается в сравнении времени выполнения реализации алгоритма обратной трассировки лучей в случае, если коэффициент отражения равен 0, и в случае, если коэффициент отражения больше 0, от количества шахматных фигур на сцене.

\section{Время выполнения реализации алгоритма}

Замеры времени выполнения для каждого количества шахматных фигур 10 раз. 
Максимальное количество фигур -- 24, изменение количества выполнялось с шагом 2. 

В Таблице \ref{tab:time} приведены замеры времени работы (в тиках).

\begin{table}[H]
	\centering
	\caption{\label{tab:time}Замеры времени выполнения в тиках (среднее значение)}
	\scalebox{0.75}
	{
		\begin{tabular}{|r|r|r|}
			\hline \specialcell{Количество фигур \\ на сцене} & 
			\specialcell{Коэффициент отражения \\ равен 0} &	\specialcell{Коэффициент отражения \\ равен 1} \\\hline
   \num{0} & \num{36928591.6}  & \num{37165577.5}  \\\hline
\num{2} & \num{63915518.8}  & \num{74191396.7}  \\\hline
\num{4} & \num{81624786.9}  & \num{97750875.7}  \\\hline
\num{6} & \num{87910836.0}  & \num{114321536.9}  \\\hline
\num{8} & \num{103859896.7}  & \num{129753559.5}  \\\hline
\num{10} & \num{120782833.2}  & \num{171895533.1}  \\\hline
\num{12} & \num{129186274.5}  & \num{173975459.4}  \\\hline
\num{14} & \num{131764314.4}  & \num{183770128.0}  \\\hline
\num{16} & \num{138377645.3}  & \num{193480551.0}  \\\hline
\num{18} & \num{157251043.0}  & \num{229007486.8}  \\\hline
\num{20} & \num{182764486.7}  & \num{273964928.9}  \\\hline
\num{22} & \num{193834301.5}  & \num{299999849.2}  \\\hline
\num{24} & \num{211815320.6}  & \num{327839202.1}  \\\hline
		\end{tabular}
	}
\end{table}

На рисунке \ref{img:time} представлен граф зависимости времени работы (в тиках) от количества шахматных фигур на сцене.

\includeimage
{time} 
{f} 
{H} 
{0.8\textwidth} 
{Зависимость времени работы (в тиках) от количества шахматных фигур на сцене}

\section*{Вывод}
\addcontentsline{toc}{section}{Вывод}

В данном разделе была описана цель исследования по времени, технические характеристики устройства, на котором будет проводиться исследование по времени, а также приведены результаты исследования.

На основании результатов исследования времени выполнения реализованного алгоритма можно сделать вывод, что в случае коэффициента отражения отличного от 0 алгоритм обратной трассировки лучей работает медленнее, чем в случае, если коэффициент отражения равен 0.

