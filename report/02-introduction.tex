\chapter*{ВВЕДЕНИЕ}
\addcontentsline{toc}{chapter}{ВВЕДЕНИЕ}

Компьютерная графика -- совокупность методов и средств преобразования в графическую форму и из графической формы с помощью ЭВМ \cite{kurov:2023}. 
Одной из задач компьютерной графики является синтез изображения \cite{kurov:2023}, который состоит из следующих этапов.

\begin{enumerate}
	\item разработка трёхмерной математической модели синтезируемой визуальной обстановки, задание положения наблюдателя, картинной плоскости, размеров окна вывода, значений управляющих сигналов, определение операторов, осуществляющих пространственное перемещение объектов визуализации;
	\item преобразования координат объектов в координаты наблюдателя, отсечение объектов сцены по границам пирамиды отсечения (видимости), вычисление двумерных перспективных проекций объектов сцены на картинную плоскость, удаление невидимых линий и поверхностей, при заданном положении наблюдателя, закрашивание и затенение видимых объектов сцены;
	\item вывод полученного полутонового изображения на экран растрового дисплея.
\end{enumerate}

Целью данной работы является разработка программного обеспечения с пользовательским интерфейсом, реализовывающего моделирование статической сцены расстановки шахматных фигур на шахматной доске.

Для достижения поставленной цели необходимо выполнить следующие задачи.

\begin{itemize}
	\item проанализировать предметную область, рассмотреть известные подходы и алгоритмы решения задачи синтеза изображения в контексте моделирования статической сцены;
	\item спроектировать программное обеспечение;
	\item выбрать средства реализации и разработать программное обеспечение;
	\item исследовать характеристики разработанного программного обеспечения.
\end{itemize}

